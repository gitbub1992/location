\chapter{2013年9月18日}
\section{Feature Selection using Weka}
在活动识别里面,在特征向量的众多特征里面抽取出一些特别重要的特征,也就是特征提取,或者降维,那对于计算量来说都是特别有用的。甚至,我在考虑online的特征选取是不是也可做也是一个问题。

不过,如果能用weka实际体验一把的话,那不管是从理解程度上,还是对于实际操作上都会有所提高吧。好,这里就给出一个参考的链接\url{https://www.laps.ufpa.br/aldebaro/weka/feature_selection.html}。

其中觉得有一句话非常有意思:\textbf{Note that strictly speaking, PCA is not a feature selection but a feature extraction method. The new attributes are obtained by a linear combination of the original attributes. Dimensionality reduction is achieved by keeping the components with highest variance.},一下就讲了一个我没有注意过的概念,就是feature selection和feature extraction的区别。

不过看了一会儿以后,发现网页用的weka版本是3.3.4的,而我目前用的是3.6.10的,weka的类结构以及提供的算法已经有了很大的变化。所以最好的参考资料,我觉得就是安装目录底下的WekaManual了。

后续又看了一个讲Weka PCA的网页(\url{http://weka.8497.n7.nabble.com/WEKA-PCA-Eigenvalues-td12741.html}),摘录两段话:

It depends on whether you're using 
weka.attributeSelection.PrincipalComponents or 
weka.filters.unsupervised.attribute.PrincipalComponents. In the both 
cases, you can comment out the block of code that creates and configures 
the Remove filter. In the former it is in the 
buildAttributeConstructor() method and in the latter it is in the 
setup() method. 

Are you using the -C option to use the mean centered + covariance matrix 
rather than standardized + correlation matrix? I believe that Matlab 
uses the covariance matrix. 

Weka中使用PCA(Filter里面选择weka.filters.unsuperived.attribute.PrincipleComponents)对UCI HAR Dataset进行特征抽取(PrincipalComponents -R 0.95 -A 5 -M -1)之后,原先561+1个属性,变成了102+1个属性。
