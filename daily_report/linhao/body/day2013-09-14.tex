\chapter{2013年9月14日}
\section{昨日总结}
昨天受的最大刺激,就是看了Dartmouth College Andrew T. Campbell教授的CS65课程。尤其是 "\url{http://www.cs.dartmouth.edu/~campbell/cs65/lecture22/lecture22.html}"和"\url{http://www.cs.dartmouth.edu/~campbell/cs65/lab5/lab5.html}"这两节课程。

这两节课程一下子就让我明白了,如何去加速度的数据中提取特征值,如何用WEKA去做分类,并且把分类算法嵌入到Android中去。课程直接提供了相应的Android代码,在手机上安装测试之后,发现效果还是非常不错的。这是多年来,我第一次对Android程序起了兴趣。

\section{今日安排}
今天的任务主要在于特征提取的算法。

看论文是必须的,但是我在想如何看完之后,自己实现一把。这有两个方面的问题:1.如何计算得到文章中的特征值,然后把特征值存成WEKA能识别的文件;2.采用什么样的数据集来做实验。姑且带着这两个问题来看文章吧。看上几篇之后,再说。

\section{批量转码}
\begin{lstlisting}
for /r  dir_name  %i in (*.txt) do iconv.exe -f GBK -t UTF-8 %i > %~ni_utf8.txt
\end{lstlisting}

\section{WEKA基础教程}
网上稍微搜了一下WEKA教程,看到有IBM的教程,果断摘录,看看。
\begin{enumerate}
  \item \url{http://www.ibm.com/developerworks/cn/opensource/os-weka1/}
  \item \url{http://www.ibm.com/developerworks/cn/opensource/os-weka2/}
  \item \url{http://www.ibm.com/developerworks/cn/opensource/os-weka3/}
\end{enumerate}

其相对应的英文版教程是:
\begin{enumerate}
  \item https://www.ibm.com/developerworks/opensource/library/os-weka1/
  \item https://www.ibm.com/developerworks/opensource/library/os-weka2/
  \item https://www.ibm.com/developerworks/opensource/library/os-weka3/
\end{enumerate}

