\chapter{2013年9月11日}
\section{学习RADAR文章}
\subsection{学习RADAR}
\begin{enumerate}
    \item
    进行定位算法的几要点:
    \begin{enumerate}
    \item
    第一:收集多组SS样本,并且从一个基站使用一个样本均值。
    \item
    第二:为了确定位置和方向,确定用哪一组数据
    \begin{enumerate}
    \item
    一:加入离线状态下的数据进行考虑
    \item
    二:信号传播建模(develop a model that accounts for both free-space loss and loss)
    \end{enumerate}
    \item
     第三:need a metric and a search methodology to compare multiple locations and pick the one that best matches the observed signal strength
    \end{enumerate}
    \item
    对于以下三种方法:The empirical method performs significantly better than both of the other methods
    \begin{enumerate}
    \item
    First:empirical method
    \item
    Second:strongest base station
    \item
    Third:random method
    \end{enumerate}
    \item
    Unlike the basic analysis where we only considered the single nearest neighbor in signal space,we now consider k nearest neighbors, for various values of k.averaging the coordinates of the neighbors may yield an estimate that is closer to the user’s true location
\end{enumerate}