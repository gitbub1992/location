\chapter{2013年9月9日}
\section{使用github协作团队开发}
\subsection{github初入门径}
首先在github.com上面申请账号,创建工程。

\textbf{Create a new repository on the command line}
\begin{lstlisting}
touch README.md
git init
git add README.md
git commit -m "first commit"
git remote add origin https://github.com/imlinhao/location.git
git push -u origin master
\end{lstlisting}

\textbf{Push an existing repository from the command line}
\begin{lstlisting}
git remote add origin https://github.com/imlinhao/location.git
git push -u origin master
\end{lstlisting}

\subsection{github常用命令}
\begin{itemize}
\item \textbf{更新项目(新加了文件)}
\begin{lstlisting}
git add .    //`这样可以自动判断新加了哪些文件,或者手动加入文件名字`
git commit   //`提交到本地仓库`
git push origin master    //`不是新创建的,不用再`add `到`remote`上了`
\end{lstlisting}

\item \textbf{更新项目(没新加文件,只有删除或者修改文件)}
\begin{lstlisting}
git commit -a          //`记录删除或修改了哪些文件`
git push origin master  //`提交到`github
\end{lstlisting}

\item \textbf{忽略一些文件,比如*.o等}
\begin{lstlisting}
vim .gitignore     //`把文件类型加入到`.gitignore`中,保存然后就可以`git add . `能自动过滤这种文件`
\end{lstlisting}

\item \textbf{clone代码到本地}
\begin{lstlisting}
git clone https://github.com/imlinhao/location.git
 //`假如本地已经存在了代码,而仓库里有更新,把更改的合并到本地的项目`
git fetch origin    //`获取远程更新`
git merge origin/master //`把更新的内容合并到本地分支`
\end{lstlisting}

\item \textbf{撤销}
\begin{lstlisting}
git reset
\end{lstlisting}

\item \textbf{删除}
\begin{lstlisting}
git rm  * //`不是用`rm
\end{lstlisting}
\end{itemize}

